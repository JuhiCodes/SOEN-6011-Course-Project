\documentclass[12pt]{report}
\usepackage[utf8x]{inputenc}
\usepackage{graphicx}
\usepackage{gensymb}
\usepackage{algorithm}
\usepackage[noend]{algpseudocode}
\usepackage{algpseudocode}
\graphicspath{ {./images/} }
\usepackage{fancyhdr}
\newcommand{\R}{\mathbb{R}}

\title{ETERNITY : FUNCTIONS}								
\author{Juhi Birju Patel}						
\date{23 July 2022}

\makeatletter
\let\thetitle\@title
\let\theauthor\@author
\let\thedate\@date
\makeatother

\pagestyle{fancy}
\fancyhf{}
\rhead{\thetitle}
\cfoot{\thepage}

\begin{document}

\begin{titlepage}
	\centering
    \vspace*{0.5 cm}
\begin{center}    \textbf{\Large Concordia University}\\[2.0 cm]	\end{center}
	\textsc{\Large  SOEN 6011 - Software Engineering Process }\\[0.5 cm]
	\rule{\linewidth}{0.2 mm} \\[0.4 cm]
	{ \huge \textbf \thetitle}\\[0.2 cm]
	{ \huge \textbf{ Function 6 : B(x,y)}}
	\rule{\linewidth}{0.2 mm} \\[1.5 cm]

\begin{center}   {\Large Problem Solution 5}\\[2.0 cm]
\end{center}	
\begin{center}   {\Large \textbf{\theauthor}} \\[0.2 cm]
                 {\large Student ID : 40190446 }\\[0.2 cm]
                 {\large https://github.com/JuhiCodes/SOEN-6011-Course-Project}
\end{center}
	
\end{titlepage}

\tableofcontents
\pagebreak
\renewcommand{\thesection}{\arabic{section}}
\newpage
\section{Unit testing}

Testing is an important part of development as it makes the code more robust. The testing of our source code is performed using Junit.

\paragraph{}Environment for testing :
\begin{enumerate}
    \item Java Eclipse environment 
    \item JUnit 5 library
\end{enumerate}

\paragraph{}Testing Steps and procedure:
\begin{enumerate}
    \item The source code is loaded in the eclipse environment.
    \item The code is validated against all range of inputs and results are analysed.
    \item All the Junit test are ran.
\end{enumerate}

\section{Junit results analysis}

Results:
\begin{enumerate}
    \item For requirements R1,R2 to check the whether the input to the function is provided in valid range, the test results are passed.
    \item For requirement R3 R4, to check that the input to the function is always positive, the test results are passed.
    \item For requirement R5, to check that the function receives only numeric values, the test results are passed. 
    \item For requirement R6, to check that the function provides accurate outputs for similar and distinct inputs, the test results are passed.
\end{enumerate}

After careful scrutinisation of the results obtained, we can state that there are no errors present in our code and it mets all the functional requiremnets mentioned.
\newpage
\begin{thebibliography}{9}

\bibitem{junit}
Junit,
\\\texttt{https://junit.org/junit5/}

\end{thebibliography}
\end{document}
